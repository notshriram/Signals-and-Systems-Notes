%set margins according to print 

\documentclass[a4paper,12pt]{book}
\setlength{\oddsidemargin}{ 0 in}
\setlength{\evensidemargin}{ 0 in}
\setlength{\topmargin}{-0.6 in}
\setlength{\textwidth}{6.5 in}
\setlength{\textheight}{8.5 in}
\setlength{\headsep}{0.75 in}
\setlength{\parindent}{0 in}
\setlength{\parskip}{0.1 in}

%packages list

\usepackage{amsmath,amsfonts,graphicx}

%
% The following commands set up the lecnum (lecture number)
% counter and make various numbering schemes work relative
% to the lecture number.
%
\newcounter{lecnum}
\renewcommand{\thepage}{\thelecnum-\arabic{page}}
\renewcommand{\thesection}{\thelecnum.\arabic{section}}
\renewcommand{\theequation}{\thelecnum.\arabic{equation}}
\renewcommand{\thefigure}{\thelecnum.\arabic{figure}}
\renewcommand{\thetable}{\thelecnum.\arabic{table}}

%
% The following macro is used to generate the header.
%
\newcommand{\RNum}[1]{\uppercase\expandafter{\romannumeral #1\relax}}

\newcommand{\lecture}[4]{
   \pagestyle{myheadings}
   \thispagestyle{plain}
   \newpage
   \setcounter{lecnum}{#1}
   \setcounter{page}{1}
   \noindent
   \begin{center}
   \framebox{
      \vbox{\vspace{2mm}
    \hbox to 6.28in { {\bf B.Tech ECE: Signals and Systems
		\hfill \RNum{2}-\RNum{1} Semester} }
       \vspace{4mm}
       \hbox to 6.28in { {\Large \hfill Chapter #1: #2  \hfill} }
       \vspace{2mm}
       \hbox to 6.28in { {\it Lecturer: #3 \hfill Scribes: #4} }
      \vspace{2mm}}
   }
   \end{center}
   \markboth{Lecture #1: #2}{Lecture #1: #2}
   {\bf Disclaimer}: {\it These notes have not been subjected to the
   usual scrutiny reserved for formal publications.  They may be distributed
   outside this class only with the permission of the Lecturer.}
   \vspace*{4mm}
}
%
% Convention for citations is authors' initials followed by the year.
% For example, to cite a paper by Leighton and Maggs you would type
% \cite{LM89}, and to cite a paper by Strassen you would type \cite{S69}.
% (To avoid bibliography problems, for now we redefine the \cite command.)
% Also commands that create a suitable format for the reference list.
\renewcommand{\cite}[1]{[#1]}
\def\beginrefs{\begin{list}%
        {[\arabic{equation}]}{\usecounter{equation}
         \setlength{\leftmargin}{2.0truecm}\setlength{\labelsep}{0.4truecm}%
         \setlength{\labelwidth}{1.6truecm}}}
\def\endrefs{\end{list}}
\def\bibentry#1{\item[\hbox{[#1]}]}

%Use this command for a figure; it puts a figure in wherever you want it.
%usage: \fig{NUMBER}{SPACE-IN-INCHES}{CAPTION}
\newcommand{\fig}[3]{
			\vspace{#2}
			\begin{center}
			Figure \thelecnum.#1:~#3
			\end{center}
	}



% Use these for theorems, lemmas, proofs, etc.
\newtheorem{theorem}{Theorem}[lecnum]
\newtheorem{lemma}[theorem]{Lemma}
\newtheorem{proposition}[theorem]{Proposition}
\newtheorem{claim}[theorem]{Claim}
\newtheorem{corollary}[theorem]{Corollary}
\newtheorem{definition}[theorem]{Definition}
\newenvironment{proof}{{\bf Proof:}}{\hfill\rule{2mm}{2mm}}


\begin{document}
\title{\Large{\textbf{Signals and Systems}}}
\author{By Shriram R}
\date{August 2019}
\maketitle
\let\cleardoublepage\clearpage
\tableofcontents
\chapter{Signal Analysis}
%\lecture{**LECTURE-NUMBER**}{**DATE**}{**LECTURER**}{**SCRIBE**}
\lecture{1}{Signal Analysis}{Syed Munavvar Hussain}{Shriram R}
%\footnotetext{These notes are partially based on those of Nigel Mansell.}

% **** YOUR NOTES GO HERE:

Signals can be used to describe a wide range of natural phenomena. A signal is generally imagined as a pattern of variations of some quantity with respect to another independent quantity. In the following section we shall look into the definition of a signal as well as a system along with some examples

\section{Introduction} 

\begin{description}
	\item[Signal] It is defined as a function of any independent variable. Generally speaking, a signal is a function of time which conveys some sort of information.\\
\textbf{E.g;}
\begin{itemize}
\item Speech or Voice signals
\item Image signals 
\item etc
\end{itemize}
\item[System] It is a collection of objects which work together to perform a particular task\\
From a communications standpoint, systems are used to process signals.
\end{description}

\section{Classification of Signals}
If a signal is defined in terms of only one independent variable, it is called a one dimensional signal otherwise it is called a multi-dimensional signal.\\
in addition to dimensions,signals can be classified on the basis of various parameters:
\subsection*{\Large{On the basis of time $t$}}
\textbf{Continuous Time Signals}\\
A signal which is defined continuously for all values of time $t$ is called a continuous time signal. It is represented by $x(t)$ (in parentheses) and are also called analog signals.

{\bf Discrete Time Signals}\\ A signal that is defined for only specific instances of time or at discrete values of time.It is represented by $x[n]$ (in brackets).It is generally obtained by sampling an analog signal.\\

{\bf These signals can be further classified as follows:}
\subsection*{\Large{On the basis of periodicity}}
\textbf{Periodic Signals :}\bigskip\\
{\bf Continuous Time Periodic Signals }\\
a signal is said to be CT-periodic if it repeats after a certain time interval $T_0$\\
Mathematically, it is defined as the signal which satisfies :\\
$x(t)=x(t+T_0)  \forall  t  \in R$ \bigskip\\
{\bf Discrete Time Periodic Signals }\\
a signal is said to be DT-periodic if it repeats after a certain time interval $N_0$\\
Mathematically, it is defined as the signal which satisfies :\\
$x[n]=x[n+N_0]  \forall  n  \in Z$\bigskip\\
{\bf Aperiodic Signals :}\\ A signal that does not satisfy the above conditions is called aperiodic signal. It may be viewed as a limiting case of a periodic signal in which period tends to Infinity.
\subsection*{\Large{Even and Odd signals }}
	a signal $x(t)$ is said to be even if it satisfies : \\
	\begin{align*}
	x(t)=x(-t){\hspace{4mm}(CT)}\\
	x[n]=x[-n]{\hspace{4mm}(DT)}
	\end{align*}
	a signal x(t) is said to be odd if it satisfies : \\
	\begin{align*}x(t)=-x(-t){\hspace{4mm}(CT)}\\x[n]=-x[-n]{\hspace{4mm}(DT)}\end{align*}	
	if it satisfies neither it is said to be neither even nor odd.
\section{Next topic}

Here is how to define things in the proper mathematical style.
Let $f_k$ be the $AND-OR$ function, defined by

\[ f_k(x_1, x_2, \ldots, x_{2^k}) = \left\{ \begin{array}{ll}

	x_1 & \mbox{if $k = 0$;} \\

	AND(f_{k-1}(x_1, \ldots, x_{2^{k-1}}),
	   f_{k-1}(x_{2^{k-1} + 1}, \ldots, x_{2^k}))
	 & \mbox{if $k$ is even;} \\

	OR(f_{k-1}(x_1, \ldots, x_{2^{k-1}}),
	   f_{k-1}(x_{2^{k-1} + 1}, \ldots, x_{2^k}))	
	& \mbox{otherwise.} 
	\end{array}
	\right. \]
Here is a citation, just for fun~\cite{CW87}.

\section*{References}
\beginrefs
\bibentry{CW87}{\sc D.~Coppersmith} and {\sc S.~Winograd}, 
``Matrix multiplication via arithmetic progressions,''
{\it Proceedings of the 19th ACM Symposium on Theory of Computing},
1987, pp.~1--6.
\endrefs

% **** THIS ENDS THE EXAMPLES. DON'T DELETE THE FOLLOWING LINE:

\end{document}





